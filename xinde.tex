\documentclass[UTF8]{article}
%\usepackage{homework}
\usepackage{geometry}
\geometry{a4paper,scale=0.8}
\usepackage[table,xcdraw]{xcolor}
\usepackage{graphicx} %插入图片的宏包
\usepackage{float} %设置图片浮动位置的宏包
\usepackage{subfigure} %插入多图时用子图显示的宏包
\usepackage{enumerate}
\usepackage{fancyhdr}  % header,footer的设置
% \usepackage{extramarks}
\usepackage{amsmath}  % 数学公式
\usepackage{amsthm}
\usepackage{amsfonts}
\usepackage{tikz}  % 绘图
\usepackage{algorithm}  % 算法
\usepackage{algorithmicx}
\usepackage{algpseudocode}  % 伪代码
\usepackage[UTF8]{ctex}  % 支持中文
%\title{我的份}
\author{19041822 罗汉东}
\title{
	\includegraphics[scale = 1.0]{HDU.png}\\
    \vspace{1in}
    \textmd{ \Huge\textbf{创业基础}}\\
    \textmd{\textbf{学习笔记与学习心得}}\\
   	\vspace{4.5in}
}
\begin{document}
\maketitle
\newpage
\tableofcontents
\newpage
\section{第一章\ 创业概念}
创业是一个从零到一、从无到有、从弱到强的过程,需要一定程度的机会获取,资源整合和团队合作。
\paragraph{创业的选择}
\begin{itemize}
    \item 第一个选择是选择一条跑道,或者说选择什么样的竞争。
    \item 第二个选择是我的顾客是谁。
    \item 第三个问题叫选择技术,或者是选择竞争优势的来源。
\end{itemize}
一个成功的企业家,第一要有想法,第二要有说法,第三要有做法。思考能力,表达能力,执行能力,三个重要的能力都同时具备的时候,才能成为一个成功的领导者。

很多成功的企业家,都有一个共同的特点,就是他们都有强烈的企业家精神。有一句话叫做:"成大事者不拘小节,成熟者不拘小节"。所以,他们能够以身作则,带领整个集团开疆拓土,把自己的企业做得风生水起。

我们要学会换位思考。在我们企业内部,有一个特别好的现象,就是员工和企业的矛盾,大家都认为自己是老板,老板就是要听自己的,这实际上就是一种矛盾体现。在企业内部,一切的矛盾都是人的矛盾、是利益的矛盾、是思想的矛盾、是技术的矛盾、是经营的矛盾。但是,我们在企业外面,是不是也存在这样一种矛盾?我们在外面,员工和我们之间存在一个什么样的矛盾呢?这个矛盾实际是一个人的问题。

如果我们去研究这个问题的时候,我们发现,人和企业之间的矛盾实际上是人际关系的矛盾,是利益的矛盾,是思想的矛盾,是利益的矛盾,是技术的矛盾,是经营的矛盾,是人的问题。我们要学会换位思考。当你的员工和你的企业之间发生矛盾的时候,你不要着急和员工去争吵,更不能和员工发生冲突,而是要冷静地去解决他。

这样一种做法你会觉得很好,因为这种心态会使员工和企业之间的关系变得非常的亲密无间。

想来也非常有道理,有一个创新的想法,是创业的基础,但是有表达能力,才能更好的说服他人说服团队,有执行力才是保障创业成功的基石。本周的学习让我对创业有了更进一步的了解。


\section{第二章\ 创业起点}
\paragraph{创业机会}
创业本身是个识别和把握机会并创造出市场新价值的过程。为了搜寻这种机会,创业家需要贡献出自己的财力、精力、能力以及宝贵的时间,同时要承担相应的财产的、
社会的和意志上的风险。

风险和收益成正比。如果你是风险投资,你就会得到比投入其他行动要少得多的回报,风险投资的回报是在你的投资组合中产生的。而在创业中,不管你是初创企业还是大公司,你都是处于风险的境地。如果你的公司成长很快,并且你的业务有很大的增长空间的话,你就会感受到压力了。你的压力是来自于你的投资组合,你的风险投资组合是有限的。你需要承担的风险是你的投入,以及你的收入。

如果你是新公司的创始人,你会面临很多风险,比如说公司的资本结构问题,你的公司的管理和运营问题,还有就是公司资金的问题。在这一系列的风险里,风险投资家能够帮你渡过一些难关,如果你不是风险投资家的话。

但是,如果你是风险投资家的话,我觉得你的压力会更大,因为如果你不是风险投资家的话,你需要承担的风险会更多。在一个新创企业中,风险投资家所处的位置对企业的发展会起到很大的作用。风险投资家需要把自己投身到企业创业的整个过程当中,这是他们工作的一部分,并且为了实现目标,需要承受风险和压力。

\paragraph{创业团队应该做到:}
\begin{description}
    \item[知己知彼]团队中所有成员都应该相互熟悉、知根知底。
    \item[才华各异]创业团队应该是成员各有所长,相互补充,相得益彰。
    \item[单一核心]带头人作为创业团队中的核心人物,应是团队成员在合作共事过程中发自内心认可的,有远见、威望、魄力和决断力的人
\end{description}
\paragraph{创业的两层含义:}
\begin{description}
\item[活动]主要指创业者及其团队培育和创建新企业或新事业而采取的行动(包括新组织的生存和初期发展)。
\item[精神]主要指创业者及其团队在开展创业活动中所表现出来的抱负、执着、坚忍不拔和创新等品质。
\end{description}
\section{第三章\ 创业过程}
团队领导制定计划后要组织团队成员去实施,并根据既定目标不断跟踪和修正,以达到理想的目标或业绩。同伴之间合作一定要理性,要严格对待合作协议,相互信任,同甘共苦,否则将产生巨大的矛盾,致使合作破裂。
\paragraph{创业机会评价的特殊性}
\begin{itemize}
    \item 对机会的评价首先来自于初始判断:“假设加上简单计算”
    \item	对机会价值的进一步评价需要依靠调查研究,对机会价值做进一步评价
    \item	在预测分析、调查论证的基础上开展市场预测,是创业者的必修课程
\end{itemize}
创业陷阱与风险何在?
\begin{enumerate}[1)]
\item 技术和产品不成熟
\item 观念偏差,机会转瞬即逝
\item 机会的力量是一种偶然因素
\item 进入障碍低,竞争对手既多且强
\end{enumerate}
\paragraph{创业的融资过程可以大致分成以下几点:}
\begin{itemize}
\item 做好融资前的准备
\item 积累人脉资源,了解融资渠道
\item 准备创业和计划
\end{itemize}

\section{第四章\ 创业方法论}
\paragraph{创业的要素:}
\begin{description}
\item[创业者]是创业行为的主体,是具有开拓精神和商业头脑的开创者。这里所说的创业不是指从事简单的生产经营活动,而是指能够创造价值的一定是创业。创业是创新和冒险的代名词,也是创业理想的实现。创业是指以个人或家庭的力量,通过自己的努力,为了某一目标,通过自觉的行动,克服生活工作中面临的困难,在社会上寻求生存和发展,为自己创造财富的活动,主要分为两种形态:1.单纯的创业,是指个人或家庭的力量,通过自己的努力,为了某一目标,通过自觉的行动,克服生活工作中面临的困难,在社会上寻求生存和发展,为自己创造财富的活动;2.多元化的创业,是指个人或家庭的力量,通过自己的努力克服生活中面临的困难,在社会上寻求生存和发展,为自己创造财富的活动。
\item[商业机会]就是创业机会,利用这种商机,是创业者进行创业的主要驱动力量。在市场经济中,商机是每个人都可以获得的,但是商机是一种机会,商机有着广阔的发展前景,它会帮助创业者实现自己的创业理想,所以,创业者一定要懂得商机。商机是无处不在的,只要你善于抓住商机,就能获得商机带来的财富。
\item[资源]组织中的各种投入,包括人、财、物。一般来说,投入资源的主要项目有:基础工作:如资源调查、收集资料、实地考察等,基础工作包括:收集资料、整理资料、整理数据、制订计划。基础工作主要包括:建立资料收集、分析、处理系统、编制报表等。资源管理:资源管理的主要工作,包括:确定资源的数量和质量、分配物品、编制计划等。资源管理的主要工作包括:制定标准、制定检测标准、制定标准的检测方法、制定统计技术等。资源管理的主要内容包括:确定资源的使用范围、分配资源的数量、计算各类资源的质量、建立资源的系统。资源管理的主要工作包括:实施规划、建立资源的系统、制定资源的计划,实行预算,使得资源得到充分的利用。
\item[组织]协调创业活动的有机系统,也是创业的载体,创业活动的系统化、组织化、社会化要求创业活动具有高度的组织化,创业活动的成功与否、效益的好坏以及组织化程度的高低,都直接关系到整个创业活动的成败。我国高校的创业实践教育,要求高校的创业教育不仅要传授创业知识和技能,还要培养创业精神。创业精神是创业者在创业实践中表现出来的精神状态,是创业者在创业过程中自发形成的。创业者应当以高度的创业精神来面对创业中遇到的各种困难和挫折,只有这样才能成功。

创业精神不仅是创业者成功的精神支柱,也是创业实践的动力系统,它是创业者在创业过程中持之以恒地克服困难、战胜挫折并取得优异成绩的有力保证。创业者应具备高度的创新精神,具有敢于冒险、勇于尝试、善于捕捉机遇、不畏困难的创新精神;具有高度的创业能力,具有开拓创新、迎难而上的顽强性格;具有高度的创业心态,具有战略上藐视风险、战术上重视战略、思想上重视思想的创业心态;具有高度的创新思维,善于运用新理念、新观点、新方法,不断进行创造性的开创,不墨守成规,敢于创新。
\end{description}

\section{第五章\ 不一样的创业}
\paragraph{创业的基本要素包括:}
\begin{enumerate}[1)]
\item 为什么这是一个有价值的创业机会?
\item 新产品或服务要卖给谁?
\item 如何开发、生产、销售新产品或服务?尤其是如何应对现存和未来竞争的总体计划是什么?
\item 创业者是谁?即他们拥有开发创意并经营新企业所需的知识、经验和技能吗?
\item 如商业计划书筹资,则需筹多少资金需以何种方式筹资,如何使用资金以及如何实现投资收益?
\end{enumerate}
\paragraph{创业计划书的核心内容包括:}
\begin{enumerate}[1)]
    \item 执行摘要
    \item 产品(服务分析)
    \item 市场分析
    \item 营销计划
    \item 团队及组织结构
    \item 运营计划
    \item 融资计划
\end{enumerate}
\paragraph{创业计划书的核心内容包括:}
\begin{description}
\item [执行摘要]主要产品和业务范围、市场概貌、营销策略、销售计划、生产管理计划、管理者及其组织、财务计划、资金需求状况等
\item [产品(服务分析)]解决的问题、用户可以从中获得的好处、产品的优缺点、企业的保护措施、企业利润来源、企业产品升级计划
\item [市场分析] 需求预测、竞争分析、市场现状、竞争厂商、目标顾客和目标市场、市场定位等
\item [营销计划]市场机构营销渠道的选择、营销队伍、价格决议、促销计划
\item[团队及组织结构] 互补性人才、专业人才等
\end{description} 
\paragraph{分享行为实质上是闲置或者过剩资源的使用权转让。}
\begin{description}
 \item[宏观角度]共享经济可以提升全社会范围的资源使用效率;
 \item[微观角度]则可以降低顾客的交易、购买和使用成本,并且资源拥
有者可以由此获得收益。
\end{description}
\newpage
\section{学习心得}
大学生创业基础课程,是一门对当代的大学生非常重要的课程。对于许多大学生来讲,创业并不陌生,许多大学生在就读期间已经或多或少接触了一些,通过观看《创业基础》的学习课程,我明白了许多东西,原来创业不是说创就能创的,创业是一项充满挑战的事业、是一个很漫长而又艰难的过程,需要进行积极的准备和挑战,才能有所成就。

首先我们必须要了解如何选择我们的赛道,选择合适的竞争对手,选择提供何种技术,选择要提供的服务,对于刚毕业的大学生来说,由于刚步入社会不久,还没有习惯社会的丛林法则,非常容易吃亏。在面对困难和挑战时需要创业者稳住心态,冷静面对。需要学习相对应的方法论和知识

我们也需要对创业路上的风险与陷阱进行预测和防范,防止自身的损失。这要求我们要实时做好分析,防止自身观念和思考出现偏差,

创业过程大致划分为四个主要阶段,分别是机会识别、资源整合、创办新
企业、新企业生存和成长,而机会识别是我认为最重要的一步。敏锐的洞察力能
使创业者有独树一帜的预见能力,能使创业者能有效地预测和把握方向。在创业
过程中,可能你会发现创业的成功与否和知识的累计程度并不成正比,有时对于
机会的识别往往是成功的关键。发现商机,抢先占领市场份额,为创业打下基础,
走出创业成功的一步。 

在刚开始创业的起点,我们需要把握好自己的资源,把握好自己的资金,把握好自己的机会,我们也需要创建并且运营好自己的团队,对于创业团队来说,我们应该做到团队中所有成员都应该相互熟悉、知根知底。团队里的人都应该各有所长,相互补充,互相弥补各自的短板,以达到最大的创业效益。作为团队的领导者,我们应该做到让团队中的人都有相同的观点,以达到最大的创业效益。同时我应该做到有远见,有魄力,能够带领整个团队乘风破浪。

在创业过程中,可能你会发现创业的成功与否和知识的累计程度并不成正比,有时对于
机会的识别往往是成功的关键。发现商机,抢先占领市场份额,为创业打下基础,要重视团队建设,如果没有重视团队的建设,随之而来的很可能就是矛盾、猜忌与争吵,最终甚至反目成仇。创建高效团队需要全体成员各就其位、各司其职、同时更要密切配合,发挥整体效能。一个好的团队成员应该是一个懂得关心他人、热情帮助他人、勇于承担责任、懂得感恩回报的人。

如果一个人的一言一行都是为别人着想,只会让自己的团队成员看不到自己的付出,也不可能成为一个优秀的团队成员。团队里每一个人的努力都是为了团队能更好的发展壮大,只有团队成员之间的互相理解、彼此信任,才可以在团队中达成一个共识,才可以让整个团队能够更好、更稳定的发展。

创业,不是一两个人的事,需要的是团队合作精神,所以如果一个人一味的想自己独吞,那只会让团队成员都看不起你。在这个竞争激烈的团队里,如果我们能够团结一心、同舟共济,那么我们就是一个优秀的团队。创建高效团队不是一个人的事,而是一个群体、是组织。团队的建设是需要一群人的配合。如果一个人没有配合好团队,那团队的合作是很难发挥作用的,一个没有合作的团队必定是一盘散沙。每一个团队都应该有一些优秀的、值得大家学习的优秀团队文化。

创业的目的就是为了让整个团队的每一个成员都能够在一个平等的环境中实现自己的价值。每一个优秀团队都是创建高效团队的基础,所以创建一个优秀的团队,就必须要有一个好的团队文化。

合作不是成员能力简单的累加,而是通过更好的配合,发挥出化学反应,创造出
更大的价值。  如果只是简单的合作,不考虑彼此的需求,只考虑自身优势,结果必然是两败俱伤,甚至还会伤害到双方的利益。这是一个最好的现象,也是为什么在合作中要保持一定的沟通频率,因为每次的合作都是双赢的。沟通是一个相互影响的过程,要保持沟通的频率。在合作中,如果一方不能主动沟通,那么就会导致对方的不满,从而进行抵触。而沟通的时间,要以双方都满意为准,如果你说了不满意的话,对方可能会很不爽。沟通的频率要保持。

沟通时间长了,可能双方就会产生一种隔阂的心理。沟通的效果要达到双赢的结果。沟通不能只是简单的问候,还应该给予对方一些反馈。如果对方在工作中不能及时给予反馈,可以把反馈的意见记录下来,并及时和相关人员沟通。沟通是一个双向的过程,可以相互成长。在合作中,沟通的效果要达到双赢。

在沟通中,不能只是简单的问候,而应该给予对方一些反馈,这样可以及时帮助对方成长,也可以让对方看到你的用心和努力。沟通是一项长期的工作,而一份好的合作是可以持续一辈子的。

创业是一个很长的过程,既是一个完善自己的过程,也是一个融入社会、认知社会,把握行业动脉的过程。这次的《创业基础》课,让我也懂得了许多社会上的道理,为将来步入社会作了心理准备。

\end{document}