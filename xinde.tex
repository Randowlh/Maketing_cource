\documentclass[UTF8]{article}
%\usepackage{homework}
\usepackage{geometry}
\geometry{a4paper,scale=0.8}
\usepackage[table,xcdraw]{xcolor}
\usepackage{graphicx} %插入图片的宏包
\usepackage{float} %设置图片浮动位置的宏包
\usepackage{subfigure} %插入多图时用子图显示的宏包
\usepackage{enumerate}
\usepackage{fancyhdr}  % header,footer的设置
% \usepackage{extramarks}
\usepackage{amsmath}  % 数学公式
\usepackage{amsthm}
\usepackage{amsfonts}
\usepackage{tikz}  % 绘图
\usepackage{algorithm}  % 算法
\usepackage{algorithmicx}
\usepackage{algpseudocode}  % 伪代码
\usepackage[UTF8]{ctex}  % 支持中文
%\title{我的份}
\author{19041822 罗汉东}
%\begin{document}
%\maketitle
\title{
	\includegraphics[scale = 1.0]{HDU.png}\\
    \vspace{1in}
    \textmd{ \Huge\textbf{创业基础}}\\
    \textmd{\textbf{学习笔记与学习心得}}\\
   	\vspace{4.5in}
	% \textmd{19041822 罗汉东}\\
	% \textmd{班级:\qquad}\\
	% \textmd{指导老师:\qquad}\\
}
\begin{document}
\maketitle
\newpage
\tableofcontents
\newpage
\section{第一章\ 创业概念}
创业是一个从零到一、从无到有、从弱到强的过程,需要一定程度的机会获取,资源整合和团队合作。
\paragraph{创业的选择}
\begin{itemize}
    \item 第一个选择是选择一条跑道,或者说选择什么样的竞争。
    \item 第二个选择是我的顾客是谁。
    \item 第三个问题叫选择技术,或者是选择竞争优势的来源。
\end{itemize}
一个成功的企业家,第一要有想法,第二要有说法,第三要有做法。思考能力,表达能力,执行能力,三个重要的能力都同时具备的时候,才能成为一个成功的领导者。


想来也非常有道理,有一个创新的想法,是创业的基础,但是有表达能力,才能更好的说服他人说服团队,有执行力才是保障创业成功的基石。本周的学习让我对创业有了更进一步的了解。


\section{第二章\ 创业起点}
\paragraph{创业机会}
创业本身是个识别和把握机会并创造出市场新价值的过程。为了搜寻这种机会,

创业家需要贡献出自己的财力、精力、能力以及宝贵的时间,同时要承担相应的财产的、
社会的和意志上的风险。
\paragraph{创业团队应该做到:}
\begin{description}
    \item[知己知彼]团队中所有成员都应该相互熟悉、知根知底。
    \item[才华各异]创业团队应该是成员各有所长,相互补充,相得益彰。
    \item[单一核心]带头人作为创业团队中的核心人物,应是团队成员在合作共事过程中发自内心认可的,有远见、威望、魄力和决断力的人
\end{description}
\paragraph{创业的两层含义:}
\begin{description}
\item[活动]主要指创业者及其团队培育和创建新企业或新事业而采取的行动(包括新组织的生存和初期发展)。
\item[精神]主要指创业者及其团队在开展创业活动中所表现出来的抱负、执着、坚忍不拔和创新等品质。
\end{description}
\section{第三章\ 创业过程}
团队领导制定计划后要组织团队成员去实施,并根据既定目标不断跟踪和修正,以达到理想的目标或业绩。同伴之间合作一定要理性,要严格对待合作协议,相互信任,同甘共苦,否则将产生巨大的矛盾,致使合作破裂。
\paragraph{创业机会评价的特殊性}
\begin{itemize}
    \item 对机会的评价首先来自于初始判断:“假设加上简单计算”
    \item	对机会价值的进一步评价需要依靠调查研究,对机会价值做进一步评价
    \item	在预测分析、调查论证的基础上开展市场预测,是创业者的必修课程
\end{itemize}
创业陷阱与风险何在?
\begin{enumerate}[1)]
\item 技术和产品不成熟
\item 观念偏差,机会转瞬即逝
\item 机会的力量是一种偶然因素
\item 进入障碍低,竞争对手既多且强
\end{enumerate}
\paragraph{创业的融资过程可以大致分成以下几点:}
\begin{itemize}
\item 做好融资前的准备
\item 积累人脉资源,了解融资渠道
\item 准备创业和计划
\end{itemize}

\section{第四章\ 创业方法论}
\paragraph{创业的要素:}
\begin{description}
\item[创业者]是创业行为的主体,是具有开拓精神和商业头脑的开创者。
\item[商业机会]就是创业机会,利用这种商机,是创业者进行创业的主要驱动力量。
\item[资源]组织中的各种投入,包括人、财、物。
\item[组织]协调创业活动的有机系统,也是创业的载体
\end{description}

\section{第五章\ 不一样的创业}
\paragraph{创业的基本要素包括:}
\begin{enumerate}[1)]
\item 为什么这是一个有价值的创业机会?
\item 新产品或服务要卖给谁?
\item 如何开发、生产、销售新产品或服务?尤其是如何应对现存和未来竞争的总体计划是什么?
\item 创业者是谁?即他们拥有开发创意并经营新企业所需的知识、经验和技能吗?
\item 如商业计划书筹资,则需筹多少资金需以何种方式筹资,如何使用资金以及如何实现投资收益?
\end{enumerate}
\paragraph{创业计划书的核心内容包括:}
\begin{enumerate}[1)]
    \item 执行摘要
    \item 产品(服务分析)
    \item 市场分析
    \item 营销计划
    \item 团队及组织结构
    \item 运营计划
    \item 融资计划
\end{enumerate}
\paragraph{创业计划书的核心内容包括:}
\begin{description}
\item [执行摘要]主要产品和业务范围、市场概貌、营销策略、销售计划、生产管理计划、管理者及其组织、财务计划、资金需求状况等
\item [产品(服务分析)]解决的问题、用户可以从中获得的好处、产品的优缺点、企业的保护措施、企业利润来源、企业产品升级计划
\item [市场分析] 需求预测、竞争分析、市场现状、竞争厂商、目标顾客和目标市场、市场定位等
\item [营销计划]市场机构营销渠道的选择、营销队伍、价格决议、促销计划
\item[团队及组织结构] 互补性人才、专业人才等
\end{description} 
\paragraph{分享行为实质上是闲置或者过剩资源的使用权转让。}
\begin{description}
 \item[宏观角度]共享经济可以提升全社会范围的资源使用效率;
 \item[微观角度]则可以降低顾客的交易、购买和使用成本,并且资源拥
有者可以由此获得收益。
\end{description}
\newpage
\section{学习心得}
大学生创业基础课程,是一门对当代的大学生非常重要的课程。对于许多大学生来讲,创业并不陌生,许多大学生在就读期间已经或多或少接触了一些,通过观看《创业基础》的学习课程,我明白了许多东西,原来创业不是说创就能创的,创业是一项充满挑战的事业、是一个很漫长而又艰难的过程,需要进行积极的准备和挑战,才能有所成就。

首先我们必须要了解如何选择我们的赛道,选择合适的竞争对手,选择提供何种技术,选择要提供的服务,对于刚毕业的大学生来说,由于刚步入社会不久,还没有习惯社会的丛林法则,非常容易吃亏。在面对困难和挑战时需要创业者稳住心态,冷静面对。需要学习相对应的方法论和知识

我们也需要对创业路上的风险与陷阱进行预测和防范,防止自身的损失。这要求我们要实时做好分析,防止自身观念和思考出现偏差,

创业过程大致划分为四个主要阶段,分别是机会识别、资源整合、创办新
企业、新企业生存和成长,而机会识别是我认为最重要的一步。敏锐的洞察力能
使创业者有独树一帜的预见能力,能使创业者能有效地预测和把握方向。在创业
过程中,可能你会发现创业的成功与否和知识的累计程度并不成正比,有时对于
机会的识别往往是成功的关键。发现商机,抢先占领市场份额,为创业打下基础,
走出创业成功的一步。 

在刚开始创业的起点,我们需要把握好自己的资源,把握好自己的资金,把握好自己的机会,我们也需要创建并且运营好自己的团队,对于创业团队来说,我们应该做到团队中所有成员都应该相互熟悉、知根知底。团队里的人都应该各有所长,相互补充,互相弥补各自的短板,以达到最大的创业效益。作为团队的领导者,我们应该做到让团队中的人都有相同的观点,以达到最大的创业效益。同时我应该做到有远见,有魄力,能够带领整个团队乘风破浪。

在创业过程中,可能你会发现创业的成功与否和知识的累计程度并不成正比,有时对于
机会的识别往往是成功的关键。发现商机,抢先占领市场份额,为创业打下基础,要重视团队建设,如果没有重视团队的建设,随之而来的很可能就是矛盾、猜忌与争吵,最终甚至反目成仇。创建高
效团队需要全体成员各就其位、各司其职、同时更要密切配合,发挥整体效能,
合作不是成员能力简单的累加,而是通过更好的配合,发挥出化学反应,创造出
更大的价值。  

创业是一个很长的过程,既是一个完善自己的过程,也是一个融入社会、认知社会,把握行业动脉的过程。这次的《创业基础》课,让我也懂得了许多社会上的道理,为将来步入社会作了心理准备。

\end{document}